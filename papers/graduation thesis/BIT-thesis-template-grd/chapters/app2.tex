
\chapter{Maxwell Equations}


因为在柱坐标系下,$\overline{\overline\mu}$是对角的,所以Maxwell方程组中电场$\bf
E$的旋度

所以$\bf H$的各个分量可以写为:
\begin{subequations}
  \begin{eqnarray}
    H_r=\frac{1}{\mathbf{i}\omega\mu_r}\frac{1}{r}\frac{\partial
      E_z}{\partial\theta } \\
    H_\theta=-\frac{1}{\mathbf{i}\omega\mu_\theta}\frac{\partial E_z}{\partial r}
  \end{eqnarray}
\end{subequations}
同样地,在柱坐标系下,$\overline{\overline\epsilon}$是对角的,所以Maxwell方程组中磁场$\bf
H$的旋度
\begin{subequations}
  \begin{eqnarray}
    &&\nabla\times{\bf H}=-\mathbf{i}\omega{\bf D}\\
    &&\left[\frac{1}{r}\frac{\partial}{\partial
        r}(rH_\theta)-\frac{1}{r}\frac{\partial
        H_r}{\partial\theta}\right]{\hat{\bf
        z}}=-\mathbf{i}\omega{\overline{\overline\epsilon}}{\bf
      E}=-\mathbf{i}\omega\epsilon_zE_z{\hat{\bf z}} \\
    &&\frac{1}{r}\frac{\partial}{\partial
      r}(rH_\theta)-\frac{1}{r}\frac{\partial
      H_r}{\partial\theta}=-\mathbf{i}\omega\epsilon_zE_z
  \end{eqnarray}
\end{subequations}
由此我们可以得到关于$E_z$的波函数方程:
\begin{eqnarray}
  \frac{1}{\mu_\theta\epsilon_z}\frac{1}{r}\frac{\partial}{\partial r}
  \left(r\frac{\partial E_z}{\partial r}\right)+
  \frac{1}{\mu_r\epsilon_z}\frac{1}{r^2}\frac{\partial^2E_z}{\partial\theta^2}
  +\omega^2 E_z=0
\end{eqnarray}
