
\subsection{形状记忆聚氨酯的形状记忆机理}
%\label{sec:features}

形状记忆聚合物(SMP)是继形状记忆合金后在80年代发展起来的一种新型形状记忆材料\cite{Jiang2005Size}。形状记忆高分子材料在常温范围内具有塑料的性质,即刚性、形状稳定恢复性;同时在一定温度下(所谓记忆温度下)具有橡胶的特性,主要表现为材料的可变形性和形变恢复性。即“记忆初始态-固定变形-恢复起始态”的循环。

固定相只有物理交联结构的聚氨酯称为热塑性SMPU,而有化学交联结构称为热固性SMPU。热塑性和热固性形状记忆聚氨酯的形状记忆原理示意图如图\ref{fig:diagram}所示




\subsection{形状记忆聚氨酯的研究进展}
%\label{sec:requirements}
首例SMPU是日本Mitsubishi公司开发成功的……。

\subsection{水系聚氨酯及聚氨酯整理剂}

水系聚氨酯的形态对其流动性,成膜性及加工织物的性能有重要影响,一般分为三种类型\cite{Jiang2005Size} ,如表 \ref{tab:category}所示。

\begin{table}
  \centering
  \caption{水系聚氨酯分类} \label{tab:category}
  \begin{tabular*}{0.9\textwidth}{@{\extracolsep{\fill}}cccc}
  \toprule
    类别			&水溶型		&胶体分散型		&乳液型 \\
  \midrule
    状态			&溶解$\sim$胶束	&分散		&白浊 \\
    外观			&水溶型		&胶体分散型		&乳液型 \\
    粒径$/\mu m$	&$<0.001$		&$0.001-0.1$		&$>0.1$ \\
    重均分子量	&$1000\sim 10000$	&数千$\sim 20万$ &$>5000$ \\
  \bottomrule
  \end{tabular*}
\end{table}

由于它们对纤维织物的浸透性和亲和性不同,因此在纺织品染整加工中的用途也有差别,其中以水溶型和乳液型产品较为常用。另外,水系聚氨酯又有反应性和非反应性之分。虽然它们的共同特点是分子结构中不含异氰酸酯基,但前者是用封闭剂将异氰酸酯基暂时封闭,在纺织品整理时复出。相互交联反应形成三维网状结构而固着在织物表面。
……