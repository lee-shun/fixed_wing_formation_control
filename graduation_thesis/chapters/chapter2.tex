\chapter{无人机动力学模型的建立}
\label{chap:formation_dynamic_equ}
本章基于无人机编队的领从方法(leader-follower method)建立无人机编队的相对运动方程。为了与无人机自动驾驶仪的解耦控制方法相匹配,本文将无人机
运动分为水平平面运动以及竖直平面运动
内分别建立数学模型。由于无人机尺寸小,强度大,飞行包线较小,现做如下假设:
\begin{itemize}
    \item 无人机为具有6自由度的三维空间运动刚体。
    \item 忽略地球自转,将地球作为惯性系。
    \item 忽略地球曲率,即所谓的“平板地球假设”。\cite{Wusentang2013}
    \item 由于内环控制率以无人机协调(倾斜)转弯(STT)为基础(详见第\ref{chap:hardware}章),飞机满足无侧滑条件,侧滑角为0;即空速方向与机体系$O_bx_b$在同一竖直平面内。
    \item 由于编队飞行时的区域较小,领机与从机的大气环境以及地球重力场等因素完全一致。
\end{itemize}
本章中涉及的坐标系有:
\begin{enumerate}
    \item 地面坐标系$O_gx_gy_gz_g$:$O_g$点选为无人机解锁时的位置,$O_gx_g$轴指向北,$O_gy_g$轴指向东,$O_gz_g$轴符合右手定则,指向下。
    \item 导航坐标系$NED(north-east-down)$:原点选做飞机质心,$N$轴指向北,$E$轴指向东,$D$轴符合右手定则,指向下。
    \item 航迹坐标系$O_kx_ky_kz_k$:$O_k$选为无人机质心,$O_kx_k$轴始终与无人机地速方向一致,$O_kz_k$轴位于包含$O_kx_k$轴的竖直平面内,
    $o_ky_k$符合右手定则,指向右。
    \item 机体坐标系$O_bx_by_bz_b$:$O_b$选为无人机质心,$O_bx_b$位于无人机的对称平面内,平行于机身轴线或者机翼的平均气动弦线,指向前;$O_bz_b$亦在对称平面之内,
    垂直于$O_bx_b$轴,指向下;$O_by_b$垂直于对称平面,指向右。机体坐标系始终与无人机固连。
    \item 气流坐标系$O_ax_ay_az_a$:气流坐标系又被称作风坐标系或者速度坐标系;$O_a$取作无人机质心,$O_ax_a$始终指向无人机的空速方向;$O_az_a$位于无人机对称面之内,
    垂直于$O_ax_a$轴,指向下;$O_ay_a$轴垂直于$O_ax_az_a$平面,指向右。只有在大气风速$V_{wind}=0$时,航迹系的$O_kx_k$才与气流坐标系的$O_ax_a$重合。
\end{enumerate}
本章中领机与从机的各运动学量以及几何关系分别用上标$l,f$标记,从机期望值以$“des”$上标标记;
运动学量以及几何关系所属坐标系关系则用各个坐标系的字母作为下标标记。
领机、从机在水平以及竖直平面内的几何关系分别在图\ref{fig:c02-2d_level_motion}和图\ref{fig:c02-2d_vert_motion}给出;
\begin{figure}[H]
    \centering
    \includegraphics[width=0.75\textwidth]{figures/c2/2d_level_motion.png}
    \caption{水平平面双机编队几何关系}\label{fig:c02-2d_level_motion}
\end{figure}
\begin{figure}[H]
    \centering
    \includegraphics[width=0.75\textwidth]{figures/c2/2d_vert_motion.png}
    \caption{竖直平面双机编队几何关系}\label{fig:c02-2d_vert_motion}
\end{figure}
在图\ref{fig:c02-2d_level_motion}中:
$(x_{g}^l,y_{g}^l),(x_{g}^f,y_{g}^f),(x_{g}^{des},y_{g}^{des})$分别为领机、从机以及从机期望编队位置在地面坐标系$O_gx_gy_g$平面之中的分量;
$\Psi^l,\Psi^f$分别为领机与从机的偏航角(yaw angle);
$\chi^l,\chi^f$分别为领机与从机的航迹偏角(航迹方位角);
$V_a,V_{wind},V_g$分别为领机和从机的空速、风速以及地速向量;
$a_{b}^{des}$是从机产生的、在体轴系下的期望的法向加速度。

在图\ref{fig:c02-2d_vert_motion}中:
$z_{g}^l,,z_{g}^f,z_{g}^{des}$分别为领机、从机以及从机期望编队位置在地面坐标系$O_gz_g$轴上的分量;
$\theta^l,\theta^f$分别为领机与从机的俯仰角(pitch angle);
$\gamma^l,\gamma^f$分别为领机与从机的航迹倾角(航迹倾斜角);
$V_a,V_{wind},V_g$分别为领机和从机的空速、风速以及地速向量;

在图\ref{fig:c02-2d_level_motion}和图\ref{fig:c02-2d_vert_motion}中,由飞机飞行动力学可知,从机与领机三维运动学方程均为:
\begin{equation}
    \left\{
    \begin{array}{l}
        \frac{dx_g}{dt}=V_g\cos\gamma\cos\chi\\
        \frac{dy_g}{dt}=V_g\cos\gamma\sin\chi\\
        \frac{dz_g}{dt}=-V_g\sin\gamma
    \end{array}
    \right .
    \label{fol_motion_eauation}
\end{equation}
现考虑无风情况下,则由图\ref{fig:c02-2d_level_motion}可知,无人机航迹偏角等于航向角,即$\Psi=\chi$;无人机在平衡状态下,迎角很小(本无人机约在2.3°左右),由图\ref{fig:c02-2d_vert_motion}可得$\theta\approx\gamma$。
于是方程组\ref{fol_motion_eauation}可改写为:
\begin{equation}
    \left\{
    \begin{array}{l}
        \frac{dx_g}{dt}=V_g\cos\theta\cos\Psi\\
        \frac{dy_g}{dt}=V_g\cos\theta\sin\Psi\\
        \frac{dz_g}{dt}=-V_g\sin\theta
    \end{array}
    \right .
    \label{fol_motion_eauation1}
\end{equation}
方程组\ref{fol_motion_eauation1}的第1、2两式表示无人机在水平平面内的运动轨迹;第3式表示无人机在竖直平面内的运动轨迹。方程组中,控制的直接输入量为从机的$V_{g}^f,\theta^f,\Psi^f$,再确定飞机的初始运动量之后,可唯一确定领机与从机的
运动规律。

值得注意的是:上述控制量并不能直接由编队控制器产生,但经过理想内环控制器以及无人机动力学模型之后,将产生相应的上述的直接控制量,完整流程将在第\ref{chap:controller_design}章中介绍。
